\documentclass[11pt]{article}

\usepackage[utf8]{inputenc}
\usepackage[spanish]{babel}
\usepackage{amsmath}
\usepackage{a4wide}
\usepackage{graphicx}
\usepackage{minted}

\title{Métodos Numéricos II - Cuerda vibrante}
\author{Unai Aguilera Irazabal\\ DNI: 45663055M}

\begin{document}
\maketitle
\tableofcontents

\section{Introducción}
La ecuación que define el movimiento de la cuerda vibrante amortiguada es la siguiente

\begin{equation}
\frac{\partial^2 y}{\partial{t^2}} = \frac{Tg}{\rho}\frac{\partial^2 y}{\partial{x^2}} 
	- B\frac{\partial y}{\partial{t}}
\label{eq:cuerda}
\end{equation}

donde $B$ es la magnitud de la fuerza de amortiguamiento, mientras que T es la tensión de
la cuerda, $g$ la aceleración de la gravedad y $\rho$ su densidad lineal. 
La ecuación de la cuerda vibrante es una ecuación de tipo hiperbólico que para su resolución
es necesario conocer los valores de frontera (posición y/o velocidad de los extremos) y
dos condiciones iniciales, los valores de $y$ y la velocidad de cada punto en el instante
inicial $t=0$.

En el caso del problema propuesto, las condiciones iniciales son

\begin{subequations}
\begin{flalign}
	&y(x)|_{t=0} = \frac{x}{3},~~~~~~~~~~~~~~~~~~ 0 <= x < \frac{3}{5}\\
	&y(x)|_{t=0} = \frac{1}{2}(1 - x),~~~~~~~~~ \frac{3}{5} <= x <= 1\\
	&\frac{\partial{y}}{\partial{t}}|_{t=0} = x(x-1)
\end{flalign}
\end{subequations}

\section{Método de solución}
La ecuación puede ser resuelta de forma numérica mediante la substitución de las derivadas
por aproximaciones con diferencias finitas. Así, cada una de las derivadas que aparecen
pueden substituirse por las siguientes aproximaciones

\begin{subequations}
\begin{flalign}
	&\frac{\partial^2 y}{\partial{x^2}} = \frac{y^j_{i+1} - 2y^j_i + y^j_{i-1}}{(\Delta{x})^2}\\
	&\frac{\partial^2 y}{\partial{t^2}} = \frac{y^{j+1}_i - 2y^j_i + y^{j-1}_i}{(\Delta{t})^2}\\
	&\frac{\partial{y}}{\partial{t}} = \frac{y^{j+1}_i - y^{j-1}_i}{2\Delta{t}}
\end{flalign}
\end{subequations}

Substituyendo las aproximaciones en la ecuación \ref{eq:cuerda}, se obtiene la siguiente
expresión

\begin{equation}
\frac{y^{j+1}_i - 2y^j_i + y^{j-1}_i}{(\Delta{t})^2} = 
	\frac{Tg}{\rho}\frac{y^j_{i+1} - 2y^j_i + y^j_{i-1}}{(\Delta{x})^2}
	- B \frac{y^{j+1}_i - y^{j-1}_i}{2\Delta{t}}
\end{equation}

de donde reordenando términos 

\begin{equation}
y^{j+1}_{i} - 2y^{j}_i + y^{j-1}_i = \frac{Tg(\Delta{t})^2}{\rho(\Delta{x})^2}
	(y^j_{i+1} - 2y^j_i + y^j_{i-1}) - \frac{B\Delta{t}}{2}(y^{j+1}_i - y^{j-1}_i)
\label{eq:aproximacion}
\end{equation}

Si en la expresión anterior el valor $\frac{Tg(\Delta{t})^2}{\rho(\Delta{x})^2}$ se iguala
a la unidad y se despeja el desplazamiento $y^{j+1}_i$, que resulta al final del paso de
tiempo actual, se obtiene la siguiente expresión

% \begin{equation}
% y^{j+1}_i = y^j_{i+1} + y^j_{i-1} - y^{j-1}_i - B\frac{\Delta{t}}{2}(y^{j+1}_i - y^{j-1}_i)
% \end{equation}

\begin{equation}
y^{j+1}_i = \frac{y^j_{i+1} + y^j_{i-1} + (\frac{B\Delta{t}}{2} - 1)y^{j-1}_i}{(\frac{B\Delta{t}}{2} + 1)}
\end{equation}

que permite calcular el valor de un nodo i a partir de la información de los nodos vecinos
a derecha e izquierda y del valor del propio nodo en un instante anterior. El valor de 
$\Delta{t}$ puede obtenerse a partir de la simplificación realizada anteriormente 

\begin{equation}
\Delta{t} = \frac{\Delta{x}}{\sqrt{\frac{Tg}{\rho}}}
\end{equation}

y que permite obtener el tamaño del paso de tiempo a partir de las características físicas
de la cuerda vibrante y del paso espacial definido para la subdivisión en intervalos de
diferencias finitas durante la aproximación de las derivadas.

La expresión anterior permite obtener la evolución del sistema en todo paso de tiempo
salvo en el primero ($0 \rightarrow 1$). Sin embargo, este puede determinarse a partir del 
conocimiento de la velocidad inicial del sistema proporcionada por 
$\frac{\partial{y}}{\partial{x}}|_{t=0}$ como condiciones iniciales. Así, en el paso
inicial

\begin{equation}
\frac{y^1_i - y^{-1}_i}{2\Delta{t}} = \frac{\partial{y}}{\partial{x}}|_{t=0}
\label{eq:aprox_derivada}
\end{equation}

A partir de la ecuación \ref{eq:aproximacion} y substituyendo nuevamente el valor de 
$\frac{Tg(\Delta{t})^2}{\rho(\Delta{x})^2}$ por la unidad se obtiene

\begin{equation}
y^{j+1}_i = y^{j}_{i+1} + y^{j}_{j-1} - y^{j-1}_i - \frac{B\Delta{t}}{2}(y^{j+1}_i - y^{j-1}_i)
\end{equation}

utilizando esta ecuación para el paso inicial y substituyendo en ella la aproximación dada
por la ecuación \ref{eq:aprox_derivada}, y tras despejar nuevamente $y^{j+1}_i$ se obtiene
la siguiente expresión para el paso inicial del método numérico

\begin{equation}
y^1_i = \frac{y^0_{i+1} + y^0_{i-1}}{2} + \Delta{t}\frac{\partial{y}}{\partial{x}}|_{t=0}
	- \frac{B\Delta{t}^2}{2}\frac{\partial{y}}{\partial{x}}|_{t=0}
\end{equation}

\end{document}