\documentclass[11pt]{article}

\usepackage[utf8]{inputenc}
\usepackage[spanish]{babel}
\usepackage{amsmath}
\usepackage{a4wide}
\usepackage{graphicx}
\usepackage{minted}

\title{Métodos Numéricos II - Cuerda vibrante}
\author{Unai Aguilera Irazabal\\ DNI: 45663055M}

\begin{document}
\maketitle
\tableofcontents

\pagebreak
\renewcommand{\tablename}{Tabla}

\section{Introducción}
La ecuación que define el movimiento de una cuerda vibrante amortiguada es la siguiente

\begin{equation}
\frac{\partial^2 y}{\partial{t^2}} = \frac{Tg}{\rho}\frac{\partial^2 y}{\partial{x^2}} 
	- B\frac{\partial y}{\partial{t}}
\label{eq:cuerda}
\end{equation}

donde $B=2.0$ es la magnitud de la fuerza de amortiguamiento, $T=10$ Kg su tensión,
$\rho = 500$ g/m su densidad lineal y $g = 9.8$ m/s la aceleración de la gravedad. 

La ecuación de la cuerda vibrante es una ecuación de tipo hiperbólico que para su resolución
es necesario conocer los valores de frontera (posición y/o velocidad de los extremos) y
dos condiciones iniciales, los valores de $y$ y la velocidad de cada punto en el instante
inicial $t=0$.
En el caso del problema propuesto, las condiciones iniciales son

\begin{subequations}
\begin{flalign}
	&y(x)|_{t=0} = \frac{x}{3},~~~~~~~~~~~~~~~~~~ 0 <= x < \frac{3}{5}\\
	&y(x)|_{t=0} = \frac{1}{2}(1 - x),~~~~~~~~~ \frac{3}{5} <= x <= 1\\
	&\frac{\partial{y}}{\partial{t}}|_{t=0} = x(x-1)
\end{flalign}
\label{eq:condiciones_iniciales}
\end{subequations}

\section{Método de resolución}
La ecuación \eqref{eq:cuerda} puede ser resuelta de forma numérica mediante la substitución
de las derivadas por aproximaciones con diferencias finitas. Así, cada una de las derivadas
que aparecen pueden substituirse por las siguientes aproximaciones

\begin{subequations}
\begin{flalign}
	&\frac{\partial^2 y}{\partial{x^2}} = \frac{y^j_{i+1} - 2y^j_i + y^j_{i-1}}{(\Delta{x})^2}\\
	&\frac{\partial^2 y}{\partial{t^2}} = \frac{y^{j+1}_i - 2y^j_i + y^{j-1}_i}{(\Delta{t})^2}\\
	&\frac{\partial{y}}{\partial{t}} = \frac{y^{j+1}_i - y^{j-1}_i}{2\Delta{t}}
\end{flalign}
\end{subequations}

Substituyendo las aproximaciones en la ecuación \eqref{eq:cuerda}, se obtiene la siguiente
expresión

\begin{equation}
\frac{y^{j+1}_i - 2y^j_i + y^{j-1}_i}{(\Delta{t})^2} = 
	\frac{Tg}{\rho}\frac{y^j_{i+1} - 2y^j_i + y^j_{i-1}}{(\Delta{x})^2}
	- B \frac{y^{j+1}_i - y^{j-1}_i}{2\Delta{t}}
\end{equation}

de donde reordenando términos 

\begin{equation}
y^{j+1}_{i} - 2y^{j}_i + y^{j-1}_i = \frac{Tg(\Delta{t})^2}{\rho(\Delta{x})^2}
	(y^j_{i+1} - 2y^j_i + y^j_{i-1}) - \frac{B\Delta{t}}{2}(y^{j+1}_i - y^{j-1}_i)
\label{eq:aproximacion}
\end{equation}

Si en la expresión anterior el valor $\frac{Tg(\Delta{t})^2}{\rho(\Delta{x})^2}$ se iguala
a la unidad y se despeja el desplazamiento $y^{j+1}_i$, que resulta al final del paso de
tiempo actual, se obtiene la siguiente expresión

\begin{equation}
y^{j+1}_i = \frac{y^j_{i+1} + y^j_{i-1} + (\frac{B\Delta{t}}{2} - 1)y^{j-1}_i}{(\frac{B\Delta{t}}{2} + 1)}
\label{eq:pasos_siguientes}
\end{equation}

que permite calcular el valor de un nodo i a partir de la información de los nodos vecinos
a derecha e izquierda y del valor del propio nodo en un instante anterior. El valor de 
$\Delta{t}$ puede obtenerse a partir de la simplificación realizada anteriormente 

\begin{equation}
\Delta{t} = \frac{\Delta{x}}{\sqrt{\frac{Tg}{\rho}}}
\label{eq:paso_tiempo}
\end{equation}

y que permite obtener el tamaño del paso de tiempo a partir de las características físicas
de la cuerda vibrante y del paso espacial definido para la subdivisión de la cuerda en
intervalos de diferencias finitas durante la aproximación de las derivadas.

La expresión anterior \eqref{eq:pasos_siguientes} permite obtener la evolución del sistema
en todo paso de tiempo salvo en el primero ($0 \rightarrow 1$). Sin embargo, este paso
puede determinarse a partir del conocimiento de la velocidad inicial del sistema
proporcionada por $\frac{\partial{y}}{\partial{x}}|_{t=0}$ como condiciones iniciales.
Así, en el paso inicial

\begin{equation}
\frac{y^1_i - y^{-1}_i}{2\Delta{t}} = \frac{\partial{y}}{\partial{x}}|_{t=0}
\label{eq:aprox_derivada}
\end{equation}

y a partir de la ecuación \eqref{eq:aproximacion} y substituyendo nuevamente el valor de 
$\frac{Tg(\Delta{t})^2}{\rho(\Delta{x})^2}$ por la unidad se obtiene

\begin{equation}
y^{j+1}_i = y^{j}_{i+1} + y^{j}_{j-1} - y^{j-1}_i - \frac{B\Delta{t}}{2}(y^{j+1}_i - y^{j-1}_i)
\end{equation}

donde para $j=0$ y substituyendo la aproximación dada por la ecuación
\ref{eq:aprox_derivada} y, tras despejar nuevamente $y^{j+1}_i$, se obtiene la siguiente
expresión para el paso inicial del método numérico

\begin{equation}
y^1_i = \frac{y^0_{i+1} + y^0_{i-1}}{2} + \Delta{t}\frac{\partial{y}}{\partial{x}}|_{t=0}
	- \frac{B\Delta{t}^2}{2}\frac{\partial{y}}{\partial{x}}|_{t=0}
\label{eq:paso_inicial}
\end{equation}

\section{Caso no amortiguado}
Para el caso de la cuerda vibrante sin amortiguamiento, las ecuaciones para su resolución
mediante el método numérico por aproximación de diferencias finitas se obtienen haciendo
$B=0$ en las ecuaciones \eqref{eq:pasos_siguientes} y \eqref{eq:paso_inicial}. Además, es
necesario llevar a cabo la substitución de $\frac{\partial{y}}{\partial{x}}|_{t=0}$ por la
expresión indicada en las condiciones iniciales del problema dadas en
\eqref{eq:condiciones_iniciales}.

\begin{subequations}
\begin{flalign}
	&y^1_i = \frac{y^0_{i+1} + y^0_{i-1}}{2} + x(x-1)\Delta{t}\\
	&y^{j+1}_i = y^j_{i+1} + y^j_{i-1} - y^{j-1}_i
\end{flalign}
\end{subequations}

Utilizando las expresiones anteriores de forma iterativa e inicializando las posiciones de
los nodos de la cuerda para $t=0$ según las condiciones indicadas en el problema
\eqref{eq:condiciones_iniciales} se obtienen los resultados recogidos en la Tabla 
\ref{tab:est_freq} para la posición de la cuerda en pasos consecutivos de tiempo, 
utilizando un espaciado de $\Delta{x} = 0.1$ cm.

En la Tabla \ref{tab:est_freq} se observa que la cuerda vuelve a su estado inicial cuando
han transcurrido 20 pasos de tiempo. Por lo tanto, teniendo en cuenta la ecuación
\ref{eq:paso_tiempo} de donde es posible obtener el valor de un paso de tiempo
$\Delta{t} = 0.0071$ s mediante la substitución de los parámetros de la cuerda y el
intervalo espacial elegido. A partir del tamaño de un paso de tiempo se obtiene que

\begin{equation}
f = \frac{1}{n \Delta{t}} = \frac{1}{20 * 0.0071} = 7.04~Hz
\end{equation}

\begin{table}
\center
\begin{small}
\begin{tabular}{ c c c c c c c c c c c c }
\hline
Paso & 0.00 & 0.10 & 0.20 & 0.30 & 0.40 & 0.50 & 0.60 & 0.70 & 0.80 & 0.90 & 1.00 \\
\hline
\hline
0 & 0.00 & 0.03 & 0.07 & 0.10 & 0.13 & 0.17 & 0.20 & 0.15 & 0.10 & 0.05 & 0.00 \\
1 & 0.00 & 0.03 & 0.07 & 0.10 & 0.13 & 0.16 & 0.16 & 0.15 & 0.10 & 0.05 & 0.00 \\
2 & 0.00 & 0.03 & 0.06 & 0.10 & 0.13 & 0.12 & 0.11 & 0.11 & 0.10 & 0.05 & 0.00 \\
3 & 0.00 & 0.03 & 0.06 & 0.10 & 0.09 & 0.08 & 0.07 & 0.06 & 0.06 & 0.05 & 0.00 \\
4 & 0.00 & 0.03 & 0.06 & 0.05 & 0.04 & 0.04 & 0.03 & 0.02 & 0.01 & 0.01 & 0.00 \\
5 & 0.00 & 0.03 & 0.02 & 0.01 & 0.00 & -0.01 & -0.01 & -0.02 & -0.03 & -0.04 & 0.00 \\
6 & 0.00 & -0.01 & -0.02 & -0.03 & -0.04 & -0.05 & -0.06 & -0.06 & -0.07 & -0.04 & 0.00 \\
7 & 0.00 & -0.05 & -0.06 & -0.07 & -0.08 & -0.09 & -0.10 & -0.10 & -0.07 & -0.03 & 0.00 \\
8 & 0.00 & -0.05 & -0.10 & -0.11 & -0.12 & -0.13 & -0.14 & -0.10 & -0.07 & -0.03 & 0.00 \\
9 & 0.00 & -0.05 & -0.10 & -0.15 & -0.16 & -0.17 & -0.14 & -0.10 & -0.07 & -0.03 & 0.00 \\
10 & 0.00 & -0.05 & -0.10 & -0.15 & -0.20 & -0.17 & -0.13 & -0.10 & -0.07 & -0.03 & 0.00 \\
11 & 0.00 & -0.05 & -0.10 & -0.15 & -0.16 & -0.16 & -0.13 & -0.10 & -0.07 & -0.03 & 0.00 \\
12 & 0.00 & -0.05 & -0.10 & -0.11 & -0.11 & -0.12 & -0.13 & -0.10 & -0.06 & -0.03 & 0.00 \\
13 & 0.00 & -0.05 & -0.06 & -0.06 & -0.07 & -0.08 & -0.09 & -0.10 & -0.06 & -0.03 & 0.00 \\
14 & 0.00 & -0.01 & -0.01 & -0.02 & -0.03 & -0.04 & -0.04 & -0.05 & -0.06 & -0.03 & 0.00 \\
15 & 0.00 & 0.04 & 0.03 & 0.02 & 0.01 & 0.01 & -0.00 & -0.01 & -0.02 & -0.03 & 0.00 \\
16 & 0.00 & 0.04 & 0.07 & 0.06 & 0.06 & 0.05 & 0.04 & 0.03 & 0.02 & 0.01 & 0.00 \\
17 & 0.00 & 0.03 & 0.07 & 0.10 & 0.10 & 0.09 & 0.08 & 0.07 & 0.06 & 0.05 & 0.00 \\
18 & 0.00 & 0.03 & 0.07 & 0.10 & 0.14 & 0.13 & 0.12 & 0.11 & 0.10 & 0.05 & 0.00 \\
19 & 0.00 & 0.03 & 0.07 & 0.10 & 0.14 & 0.17 & 0.16 & 0.15 & 0.10 & 0.05 & 0.00 \\
\hline
20 & 0.00 & 0.03 & 0.07 & 0.10 & 0.13 & 0.17 & 0.20 & 0.15 & 0.10 & 0.05 & 0.00 \\
\hline
21 & 0.00 & 0.03 & 0.07 & 0.10 & 0.13 & 0.16 & 0.16 & 0.15 & 0.10 & 0.05 & 0.00 \\
22 & 0.00 & 0.03 & 0.06 & 0.10 & 0.13 & 0.12 & 0.11 & 0.11 & 0.10 & 0.05 & 0.00 \\
23 & 0.00 & 0.03 & 0.06 & 0.10 & 0.09 & 0.08 & 0.07 & 0.06 & 0.06 & 0.05 & 0.00 \\
24 & 0.00 & 0.03 & 0.06 & 0.05 & 0.04 & 0.04 & 0.03 & 0.02 & 0.01 & 0.01 & 0.00 \\
25 & 0.00 & 0.03 & 0.02 & 0.01 & 0.00 & -0.01 & -0.01 & -0.02 & -0.03 & -0.04 & 0.00 \\
\end{tabular}
\end{small}
\caption{Resultados de la cuerda en vibración para $\Delta{x} = 0.1$ cm}
\label{tab:est_freq}
\end{table}

En la tabla \ref{tab:est_velocidad} se recogen nuevamente los resultados para algunos nodos
usando el mismo intervalo espacial $\Delta{x} = 0.1$ cm. Sin embargo, en este caso se han
añadido también las velocidades de los puntos de la cuerda. Las velocidades han sido
calculadas utilizando las diferencias de posición entre dos instantes consecutivos de
tiempo de acuerdo al siguiente esquema

\begin{equation}
\frac{\partial{y}}{\partial{t}} = \frac{y^{j + 1}_i - y^{j}_i}{\Delta{t}}
\end{equation} 

De nuevo puede observarse que la velocidad inicial es recuperada en el paso de tiempo 20,
punto en el cual comienza a repetirse el ciclo de movimiento de la cuerda.

\begin{table}[t]
\center
\begin{small}
\begin{tabular}{ c c c c c }
\hline
Paso & 0.30 & 0.40 & 0.50 & 0.60 \\
\hline
\hline
0 & 0.10 (-0.21) & 0.13 (-0.24) & 0.17 (-0.25) & 0.20 (-6.07) \\
1 & 0.10 (-0.19) & 0.13 (-0.22) & 0.16 (-6.06) & 0.16 (-6.05) \\
2 & 0.10 (-0.15) & 0.13 (-6.01) & 0.12 (-6.02) & 0.11 (-6.01) \\
3 & 0.10 (-5.92) & 0.09 (-5.95) & 0.08 (-5.96) & 0.07 (-5.95) \\
4 & 0.05 (-5.86) & 0.04 (-5.87) & 0.04 (-5.88) & 0.03 (-5.87) \\
5 & 0.01 (-5.80) & 0.00 (-5.79) & -0.01 (-5.78) & -0.01 (-5.79) \\
6 & -0.03 (-5.74) & -0.04 (-5.71) & -0.05 (-5.70) & -0.06 (-5.71) \\
7 & -0.07 (-5.68) & -0.08 (-5.65) & -0.09 (-5.64) & -0.10 (-5.65) \\
8 & -0.11 (-5.64) & -0.12 (-5.61) & -0.13 (-5.60) & -0.14 (0.22) \\
9 & -0.15 (0.21) & -0.16 (-5.59) & -0.17 (0.25) & -0.14 (0.24) \\
10 & -0.15 (0.21) & -0.20 (6.07) & -0.17 (0.25) & -0.13 (0.24) \\
11 & -0.15 (6.02) & -0.16 (6.05) & -0.16 (6.06) & -0.13 (0.22) \\
12 & -0.11 (5.98) & -0.11 (6.01) & -0.12 (6.02) & -0.13 (6.01) \\
13 & -0.06 (5.92) & -0.07 (5.95) & -0.08 (5.96) & -0.09 (5.95) \\
14 & -0.02 (5.86) & -0.03 (5.87) & -0.04 (5.88) & -0.04 (5.87) \\
15 & 0.02 (5.80) & 0.01 (5.79) & 0.01 (5.78) & -0.00 (5.79) \\
16 & 0.06 (5.74) & 0.06 (5.71) & 0.05 (5.70) & 0.04 (5.71) \\
17 & 0.10 (-0.15) & 0.10 (5.65) & 0.09 (5.64) & 0.08 (5.65) \\
18 & 0.10 (-0.19) & 0.14 (-0.22) & 0.13 (5.60) & 0.12 (5.61) \\
19 & 0.10 (-0.21) & 0.14 (-0.24) & 0.17 (-0.25) & 0.16 (5.59) \\
\hline
20 & 0.10 (-0.21) & 0.13 (-0.24) & 0.17 (-0.25) & 0.20 (-6.07) \\
\hline
21 & 0.10 (-0.19) & 0.13 (-0.22) & 0.16 (-6.06) & 0.16 (-6.05) \\
22 & 0.10 (-0.15) & 0.13 (-6.01) & 0.12 (-6.02) & 0.11 (-6.01) \\
23 & 0.10 (-5.92) & 0.09 (-5.95) & 0.08 (-5.96) & 0.07 (-5.95) \\
24 & 0.05 (-5.86) & 0.04 (-5.87) & 0.04 (-5.88) & 0.03 (-5.87) \\
\end{tabular}
\end{small}
\caption{Posición y velocidad de algunos nodos de la cuerda para $\Delta{x} = 0.1$ cm }
\label{tab:est_velocidad}
\end{table}

\subsection{Solución analítica}
\label{sec:sol_analitica}
La solución analítica en el caso de la cuerda vibrante puede obtenerse mediante el método
de separación de variables. Si en la ecuación \eqref{eq:cuerda} se realiza la siguiente
substitución

\begin{equation}
y(x,t) = u(x)v(t)
\end{equation}

es posible obtener el siguiente sistema de ecuaciones haciendo que cada parte de la ecuación
dependa únicamente de una variable e igualando cada lado a cierta constante $-K$.

\begin{subequations}
\begin{flalign}
	&\frac{v''(t)}{c^2v(t)} = -K\\
	&\frac{u''(x)}{u(x)} = -K
\end{flalign}
\label{eq:sistema_cuerda}
\end{subequations}

De acuerdo a las condiciones iniciales del problema sabemos que los extremos de la cuerda
están fijos en $u(0) = u(1) = 0$. Estas condiciones junto con la ecuación

\begin{equation}
	u''(x) + Ku(x) = 0
\end{equation}

permiten plantear un problema de valores propios cuya solución viene dada por

\begin{equation}
	K = n^2\pi^2
\end{equation}

siendo las funciones propias correspondientes

\begin{equation}
	u_n(x) = c_n\sin{n\pi{x}}
\end{equation}

donde $c_n$ son constantes arbitrarias no nulas que deben ser determinadas.

A continuación se considera la segunda ecuación de \eqref{eq:sistema_cuerda} con los
valores de K obtenidos

\begin{equation}
	v''(t) + c^2n^2\pi^2v(t) = 0
\end{equation}
 
La solución general para esta ecuación es (para cada n = 1, 2, 3)

\begin{equation}
	u(x) = c_{n,1}\cos{n\pi{c}t} + c_{n,2}\sin{n\pi{c}t}
\end{equation}

Al combinar los resultados anteriores en $y(x,t) = u(x)v(t)$ y absorber las constantes, se
obtiene para cada $n = 1, 2, 3 ...$

\begin{equation}
	y_n(x,t) = (a_n\cos{n\pi{c}t} + b_n\sin{n\pi{c}t})\sin{n\pi{x}}
\end{equation}

Sin embargo, hay que tener en cuenta que la solución completa es la combinación de todos
los posibles valores de n, por lo que finalmente

\begin{equation}
	y(x,t) = \sum\limits_{i=1}^\infty[a_n\cos{n\pi{c}t} + b_n\sin{n\pi{c}t}]\sin{n\pi{x}}
\label{eq:sol_analitica}
\end{equation}

Ahora pueden aplicarse las condiciones iniciales, que en el caso del problema son del tipo

\begin{equation}
	y(x, 0) = f(x)~~~~~~~~~~~~~~\frac{\partial{y}}{\partial{t}}(x, 0) = g(x)
\end{equation}

obteniéndose de esta forma las siguientes condiciones que permiten determinar el valor de
las constantes

\begin{subequations}
\begin{flalign}
	&\sum\limits_{i=1}^\infty a_n \sin{n\pi{x}} = f(x)\\
	&\sum\limits_{i=1}^\infty b_n n\pi{c} \sin{n\pi{x}} = g(x)
\end{flalign}
\end{subequations}

Así, para obtener el valor de las $a_n$ es necesario resolver la serie de Fourier mediante
la siguiente integral

\begin{equation}
	a_n = 2 \int_0^1 f(x) \sin{n\pi{x}}dx = 2[\int_0^\frac{3}{5} \frac{x}{3} \sin{n\pi{x}}dx
	+ \int_{\frac{3}{5}}^1 \frac{1}{2}(1-x) \sin{n\pi{x}}dx]
\end{equation}

obteniéndose mediante su solución la expresión analítica para cada $a_n$.

\begin{equation}
	a_n = 2 [\frac{\sin{n\pi{x}} - \pi{n}x\cos{n\pi{x}}}{3\pi^3n^3}\rvert_0^\frac{3}{5}
	+ [\frac{\sin(n\pi{x})}{2\pi^2n^2} + \frac{x\cos(n\pi{x})}{2\pi{n}} - \frac{\cos(n\pi{x})}{2\pi{n}}]\rvert_{\frac{3}{5}}^1]
\end{equation}

En el caso de $b_n$ se sigue el mismo procedimiento para obtener

\begin{equation}
	b_n n\pi{c} = 2 \int_0^1 g(x) \sin{n\pi{x}}dx = 2 \int_0^1 x(x-1) \sin{n\pi{x}}dx 
\end{equation}

para el que se obtiene una expresión analítica

\begin{equation}
	b_n n\pi{c} = 2[\frac{(2 - \pi^2n^2(x-1)x)\cos{n\pi{x}} + \pi{n}(2x - 1)\sin{n\pi{x}}}{\pi^3n^3}]\rvert_0^1 
\end{equation}

Finalmente, a partir de las expresiones obtenidas para $a_n$ y $b_n$ es posible calcular
la posición de la cuerda de forma analítica para cada $y(x,t)$ utilizando la expresión
proporcionada por la ecuación \eqref{eq:sol_analitica}.

El código Python que permite el cálculo de los valores analíticos para la posición de la
cuerda se muestra en la Figura \ref{fig:sol_analitica}. En todos los cálculos de la solución
analítica recogidos en este trabajo se han utilizado 100 elementos de la serie de Fourier,
tanto para $a_n$ como para $b_n$ con la finalidad de calcular la posición de la cuerda.

\subsection{Precisión del método}

Las Tablas \ref{tab:comparativa1}, \ref{tab:comparativa2} y \ref{tab:comparativa3} recogen
los resultados del cálculo de posición para diferentes intervalos $\Delta{x}$, mostrando
entre paréntesis el error $\epsilon = |y_i - y_{analitico}|$ por cada valor calculado,
utilizando para ello el resultado proporcionado para el $(x,t)$ correspondiente por la
solución analítica obtenida en la sección \ref{sec:sol_analitica}.

Como puede observarse, aun con una división en intervalos de $\Delta{x} = 0.2$ cm, se 
obtiene una buena precisión comparada con la solución analítica, ya que existe diferencia
únicamente a partir de la cuarta cifra decimal. Esta diferencia se reduce, como puede
observarse en las Tablas \ref{tab:comparativa2} y \ref{tab:comparativa3}, al incrementar
el número de intervalos en los que se divide la cuerda, lo que reduce no solamente el
valor de $\Delta{x}$ sino también el de $\Delta{t}$ que se encuentra relacionado
directamente con este. Esta reducción en el tamaño de un paso de tiempo significa que es
necesario llevar a cabo un mayor número de cálculos para alcanzar un determinado instante
de tiempo.

\begin{table}
\center
\begin{tabular}{ c c c c c }
\hline
Paso & 0.20 & 0.40 & 0.60 & 0.80 \\
\hline
\hline
0 & 0.0667 (0.0000) & 0.1333 (0.0000) & 0.2000 (0.0008) & 0.1000 (0.0000) \\
1 & 0.0644 (0.0002) & 0.1299 (0.0002) & 0.1132 (0.0002) & 0.0977 (0.0002) \\
2 & 0.0632 (0.0002) & 0.0443 (0.0004) & 0.0276 (0.0004) & 0.0132 (0.0002) \\
3 & -0.0201 (0.0002) & -0.0390 (0.0004) & -0.0557 (0.0004) & -0.0701 (0.0006) \\
4 & -0.1023 (0.0006) & -0.1201 (0.0002) & -0.1368 (0.0006) & -0.0690 (0.0002) \\
5 & -0.1000 (0.0000) & -0.2000 (0.0008) & -0.1333 (0.0000) & -0.0667 (0.0000) \\
6 & -0.0977 (0.0002) & -0.1132 (0.0002) & -0.1299 (0.0002) & -0.0644 (0.0002) \\
7 & -0.0132 (0.0002) & -0.0276 (0.0004) & -0.0443 (0.0004) & -0.0632 (0.0002) \\
8 & 0.0701 (0.0006) & 0.0557 (0.0004) & 0.0390 (0.0004) & 0.0201 (0.0002) \\
9 & 0.0690 (0.0002) & 0.1368 (0.0006) & 0.1201 (0.0002) & 0.1023 (0.0006) \\
10 & 0.0667 (0.0000) & 0.1333 (0.0000) & 0.2000 (0.0008) & 0.1000 (0.0000) \\
\end{tabular}
\caption{Precisión del método comparado con la solución analítica para $\Delta{x} = 0.2$ cm}
\label{tab:comparativa1}
\end{table}

\begin{table}
\center
\begin{tabular}{ c c c c c }
\hline
Paso & 0.10 & 0.20 & 0.30 & 0.40 \\
\hline
\hline
0 & 0.0333 (0.0000) & 0.0667 (0.0000) & 0.1000 (0.0000) & 0.1333 (0.0000) \\
1 & 0.0327 (0.0000) & 0.0655 (0.0000) & 0.0985 (0.0000) & 0.1316 (0.0000) \\
2 & 0.0322 (0.0000) & 0.0645 (0.0000) & 0.0971 (0.0000) & 0.1300 (0.0004) \\
3 & 0.0318 (0.0000) & 0.0638 (0.0000) & 0.0961 (0.0004) & 0.0871 (0.0001) \\
4 & 0.0316 (0.0000) & 0.0634 (0.0004) & 0.0538 (0.0001) & 0.0446 (0.0001) \\
5 & 0.0315 (0.0004) & 0.0216 (0.0000) & 0.0119 (0.0001) & 0.0026 (0.0001) \\
6 & -0.0100 (0.0000) & -0.0200 (0.0000) & -0.0296 (0.0001) & -0.0388 (0.0001) \\
7 & -0.0515 (0.0004) & -0.0612 (0.0000) & -0.0706 (0.0001) & -0.0796 (0.0001) \\
8 & -0.0511 (0.0000) & -0.1021 (0.0005) & -0.1112 (0.0000) & -0.1200 (0.0000) \\
9 & -0.0506 (0.0000) & -0.1011 (0.0000) & -0.1515 (0.0004) & -0.1600 (0.0000) \\
10 & -0.0500 (0.0000) & -0.1000 (0.0000) & -0.1500 (0.0000) & -0.2000 (0.0008) \\
\end{tabular}
\caption{Precisión del método comparado con la solución analítica para $\Delta{x} = 0.1$ cm}
\label{tab:comparativa2}
\end{table}

\begin{table}
\center
\begin{tabular}{ c c c c c }
\hline
Paso & 0.05 & 0.10 & 0.15 & 0.20 \\
\hline
\hline
0 & 0.0167 (0.0000) & 0.0333 (0.0000) & 0.0500 (0.0000) & 0.0667 (0.0000) \\
1 & 0.0165 (0.0000) & 0.0330 (0.0000) & 0.0495 (0.0000) & 0.0661 (0.0000) \\
2 & 0.0163 (0.0000) & 0.0327 (0.0000) & 0.0491 (0.0000) & 0.0655 (0.0000) \\
3 & 0.0162 (0.0000) & 0.0324 (0.0000) & 0.0487 (0.0000) & 0.0650 (0.0000) \\
4 & 0.0161 (0.0000) & 0.0322 (0.0000) & 0.0484 (0.0000) & 0.0646 (0.0000) \\
5 & 0.0160 (0.0000) & 0.0320 (0.0000) & 0.0481 (0.0000) & 0.0642 (0.0000) \\
6 & 0.0159 (0.0000) & 0.0319 (0.0000) & 0.0478 (0.0000) & 0.0638 (0.0000) \\
7 & 0.0159 (0.0000) & 0.0317 (0.0000) & 0.0476 (0.0000) & 0.0636 (0.0000) \\
8 & 0.0158 (0.0000) & 0.0316 (0.0000) & 0.0475 (0.0000) & 0.0634 (0.0004) \\
9 & 0.0158 (0.0000) & 0.0316 (0.0000) & 0.0474 (0.0004) & 0.0425 (0.0000) \\
10 & 0.0158 (0.0000) & 0.0316 (0.0004) & 0.0266 (0.0000) & 0.0216 (0.0000) \\
\end{tabular}
\caption{Precisión del método comparado con la solución analítica para $\Delta{x} = 0.05$ cm}
\label{tab:comparativa3}
\end{table}

\subsection{Caso con amortiguamiento}
En el caso de la cuerda con amortiguamiento es necesario utilizar las expresiones
completas dadas por las ecuaciones \eqref{eq:paso_inicial} y \eqref{eq:pasos_siguientes}
y que incluyen el parámetro $B = 2.0$.
Utilizando este esquema para la solución numérica se obtienen los resultados recogidos en
la tabla \ref{tab:est_velocidad_amortiguado}.

En este caso, como era de esperar, la cuerda no puede recuperar su posición inicial ya que
pierde energía debido a la fuerza de amortiguamiento. Así, la altura de la cuerda es 
cada vez menor, hasta que para cierto instante de tiempo detiene prácticamente su
movimiento.

Utilizando la solución numérica puede estimarse el tiempo para el que el movimiento de la
cuerda se detiene dentro de un tolerancia determinada. Se ha determinado en qué momento
cada uno de los nodos de la cuerda se encuentran en el punto $0.0$ con un velocidad $0.0$
teniendo en cuenta una tolerancia de $0.0005$ en ambos casos. Se ha comprobado que esto
sucede para un número de pasos igual a 1300, por lo que se estima que la vibración de la
cuerda se detiene para $1300 * 0.0071 = 9.23$ segundos desde su puesta en marcha.

\begin{table}
\center
\begin{small}
\begin{tabular}{ c c c c c }
\hline
Paso & 0.30 & 0.40 & 0.50 & 0.60 \\
\hline
\hline
0 & 0.10 (-0.21) & 0.13 (-0.24) & 0.17 (-0.25) & 0.20 (-6.07) \\
1 & 0.10 (-0.19) & 0.13 (-0.22) & 0.16 (-6.02) & 0.16 (-5.97) \\
2 & 0.10 (-0.14) & 0.13 (-5.92) & 0.12 (-5.89) & 0.11 (-5.92) \\
3 & 0.10 (-5.79) & 0.09 (-5.78) & 0.08 (-5.83) & 0.07 (-5.78) \\
4 & 0.05 (-5.65) & 0.05 (-5.70) & 0.04 (-5.67) & 0.03 (-5.70) \\
5 & 0.01 (-5.60) & 0.01 (-5.55) & -0.00 (-5.58) & -0.01 (-5.55) \\
6 & -0.03 (-5.46) & -0.03 (-5.47) & -0.04 (-5.42) & -0.05 (-5.47) \\
7 & -0.06 (-5.41) & -0.07 (-5.34) & -0.08 (-5.37) & -0.09 (-5.34) \\
8 & -0.10 (-5.29) & -0.11 (-5.31) & -0.12 (-5.26) & -0.13 (0.20) \\
9 & -0.14 (0.19) & -0.15 (-5.21) & -0.16 (0.23) & -0.13 (0.22) \\
10 & -0.14 (0.19) & -0.19 (5.65) & -0.16 (0.23) & -0.12 (0.22) \\
11 & -0.14 (5.56) & -0.15 (5.55) & -0.15 (5.60) & -0.12 (0.20) \\
12 & -0.10 (5.45) & -0.11 (5.51) & -0.11 (5.48) & -0.12 (5.51) \\
13 & -0.06 (5.39) & -0.07 (5.38) & -0.07 (5.43) & -0.08 (5.38) \\
14 & -0.02 (5.26) & -0.03 (5.31) & -0.04 (5.28) & -0.04 (5.31) \\
15 & 0.02 (5.21) & 0.01 (5.16) & 0.00 (5.19) & -0.01 (5.16) \\
16 & 0.05 (5.09) & 0.05 (5.10) & 0.04 (5.05) & 0.03 (5.10) \\
17 & 0.09 (-0.13) & 0.08 (4.97) & 0.08 (5.00) & 0.07 (4.97) \\
18 & 0.09 (-0.16) & 0.12 (-0.19) & 0.11 (4.90) & 0.10 (4.94) \\
19 & 0.09 (-0.18) & 0.12 (-0.20) & 0.15 (-0.21) & 0.14 (4.86) \\
20 & 0.09 (-0.17) & 0.12 (-0.20) & 0.14 (-0.20) & 0.17 (-5.25) \\
21 & 0.09 (-0.16) & 0.11 (-0.18) & 0.14 (-5.21) & 0.14 (-5.16) \\
22 & 0.08 (-0.12) & 0.11 (-5.13) & 0.11 (-5.10) & 0.10 (-5.13) \\
23 & 0.08 (-5.02) & 0.08 (-5.01) & 0.07 (-5.05) & 0.06 (-5.01) \\
24 & 0.05 (-4.90) & 0.04 (-4.94) & 0.03 (-4.92) & 0.03 (-4.95) \\
\end{tabular}
\end{small}
\caption{Posición y velocidad de la cuerda amortiguada para $\Delta{x} = 0.1$ cm}
\label{tab:est_velocidad_amortiguado}
\end{table}

En la Figura \ref{fig:sol_amortiguada} se muestra el código utilizado en este caso para el
cálculo de la solución numérica, que es similar al mostrado en Figura \ref{fig:sol_basica}
pero donde se ha modificado el cálculo del paso inicial y siguientes. En este caso
únicamente se incluye el código correspondiente a los pasos iterativos, ya que el resto
es idéntico al mostrado en la Figura \ref{fig:sol_basica}. Los cambios en las expresiones
utilizadas se encuentran marcados en el código.

\section{Conclusiones}
Se ha llevado a cabo la resolución de la ecuación de la cuerda vibrante mediante la
aplicación de un método numérico consistente en la aproximación de las derivadas mediante
diferencias finitas. Se ha obtenido un esquema iterativo que ha sido programado utilizando
el lenguaje de programación Python y que ha permitido obtener la posición y velocidad de
la cuerda para diferentes instantes de tiempo.

Por otro lado, la precisión del método numérico ha sido comparada con solución analítica,
que ha sido obtenida mediante la aplicación del método de separación de variables. Hay que
tener en cuenta que la solución analítica no es completamente exacta debido a que incluye
el cálculo de elementos de una serie de Fourier. Sin embargo, para la obtención de los
resultados analíticos se ha calculado un número alto de elementos de la serie con el fin
de conseguir una aproximación elevada en el resultado.

Durante la comparación de los resultados se ha observado que el método numérico obtiene
una buena precisión incluso para subdivisiones espaciales y temporales grandes. Como era
de esperar, la precisión puede aumentarse aumentando el número de nodos y, en consecuencia,
reduciendo el tamaño del paso de tiempo. Esto tiene como consecuencia una mayor precisión,
pero también repercute en el número de cálculos necesarios, tanto por paso, como debido
al aumento en el número de pasos necesarios para alcanzar determinado instante de tiempo.

El estudio de la cuerda vibrante se ha llevado a cabo en dos casos distintos: sin y con
amortiguación. En el caso sin amortiguación la progresión de la cuerda es periódica, y la
frecuencia de su oscilación ha podido determinarse con la búsqueda de un ciclo completo por
simple inspección de los resultados para cierto número de pasos de tiempo. Por otro lado,
en el caso de la cuerda amortiguada se ha comprobado que la amplitud de su movimiento
disminuye con el tiempo y que finalmente se detiene. Se ha estimado el intervalo necesario
para la detención de la cuerda buscando el primer paso de tiempo para el cual todos los
nodos se encuentran sobre el eje X y tienen una velocidad nula, teniendo en cuenta cierta
tolerancia para estos valores. 

Por último, en este caso el método numérico es mucho más fácil de desarrollar y utilizar
para obtener resultados que la solución numérica, que requiere la resolución de las series
de Fourier y su posterior cálculo para cierto número de coeficientes, teniendo en cuenta,
eso sí, los aspectos relacionados con la precisión del método.

\begin{figure}
\inputminted[linenos, lastline=54, fontsize=\footnotesize, tabsize=2]{python}
{../sol_analitica.py}
\caption{Código para el calculo de la solución analítica de la cuerda vibrante}
\label{fig:sol_analitica}
\end{figure}

\begin{figure}
\inputminted[linenos, lastline=56, fontsize=\footnotesize, tabsize=2]{python}
{../sol_basica.py}
\caption{Código para el cálculo de la solución numérica para el caso no amortiguado}
\label{fig:sol_basica}
\end{figure}

\begin{figure}
\inputminted[linenos, firstline=29, firstnumber=29, lastline=52, fontsize=\footnotesize, tabsize=2]{python}
{../sol_amortiguada.py}
\caption{Código para el cálculo de la solución numérica para el caso amortiguado}
\label{fig:sol_amortiguada}
\end{figure}

\end{document}